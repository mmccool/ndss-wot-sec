The economic impact of the IoT will be strongly dependent on how well devices from 
different manufacturers can interoperate.
Very often wide interoperability is taken for granted when estimating the business
benefit of IoT. 
Conversely, if IoT devices do \emph{not} interoperate,
a recent study~\cite{McK2015a} concluded that 40\% to 60\% of the 
benefit of IoT will be unattainable,
due to the inability to address use cases that cannot be satisfied by a single manufacturer.

Full interoperability is hard to achieve.
There are currently many competing IoT standards under development,
each of which is attempting to address this problem.
Most of these standards are prescriptive.
In a prescriptive standard,
devices are validated against specific requirements.
Typically the goal is that devices validated against 
a particular standard will interoperate with
other devices also validated against that standard.
It is also possible to bridge multiple standards so that
devices validated against one standard can communicate with
devices using another standard by translating communication protocols and payloads.
If one standard comes to dominate bridging will be unnecessary,
but so far such unification seems to be elusive. 
Unification is complicated by the
divergent requirements in different but overlapping IoT subdomains.
As an example of divergent requirements,
some application areas require real-time response (fixed latency responses)
and others require low power (requiring long sleep times) or wireless access
(in which it is very difficult to acheive real-time guarantees due to the
possibility of interference or collisions).
Basic interaction patterns can also vary: some standards use
a web-inspired resource-centric (RESTful) client/server interaction model,
while others use a message-centric publish-subscribe model.

%Unfortunately the prescriptive approach has some weaknesses.
In addition to the issue of the interoperability for modern prescriptive IoT standards,
there are always going to be devices that follow older standards or specifications.
There are decades-old devices in particular domains, such as building and factory
automation, that are just now being connected to the IoT.
These devices often represent major investments and cannot be economically replaced with newer,
devices conforming to the latest standard.
This is the ``brownfield'' problem.
Moreover new devices are being deployed today that have not been validated
against any particular IoT standard,
even if they use other \textit{de jure} or \textit{de facto} standards such as JSON, HTTP, or MQTT.

As an alternative to the prescriptive approach,
the W3C Web of Things (WoT) Working Group has been developing a \emph{descriptive} 
approach to IoT interoperability.
In the descriptive approach,
metadata is provided that describes how to communicate with each particular device,
using a set of interaction patterns that includes as a subset both resource-centric
and message-centric interaction models.
The metadata itself is standardized but flexible enough to describe a wide variety of
IoT network interfaces.
With this approach,
devices can but do not have to be prevalidated against 
a particular standard before being deployed.
They can be described after the fact,
and do not need any modification to be
used as part of a Web of Things system.
This solves the brownfield problem and allows
older devices as well as devices satisfying different IoT 
standards to be integrated into a unified system.  

This approach has both risks and opportunities from a security point of view.
Most obviously, IoT devices, even those conforming to a prescriptive IoT standard,
may vary widely in their support for security.
Therefore a Web of Things system
needs to manage different levels of trust for different devices.
Devices from different ecosystems or manufacturers may also take different approaches to
security, have different trust models, have different levels of acceptable risk,
and may use different security mechanisms. 
This may cause integration challenges, even if the necessary
information is provided in the metadata.

% Beyond this basic concern,
% the availability of pervasive metadata raises several other issues
% from a security perspective.
% In this paper we discuss five major issues:
% NOTE: want to use {description} here but it seems to be broken...
The goal of this paper is to set out WoT security scenarios
and start the discussion of the risks and opportunities presented
by such pervasive metadata.  
\textit{\textbf{We do not present solutions, but rather a set of problems.}}
We have identified at least five such problems:

\noindent\textbf{Local Links:}
End-point IoT devices can be deployed in many ways: 
starting with deployments in closed, only locally accessible,
networks, extending to systems on local networks but behind a proxy or a firewall 
providing access to the global internet, and ending with deployments on a globally addressed network. 
In fact, it may be possible to reach the same device via multiple paths.
As a result the metadata provided by IoT devices should allow for expression of
different ways (links) to reach each device 
and a way to securely update this information when the deployment setup changes. 
Security mechanisms with assumptions about global connectivity also may not
operate correctly in disconnected networks.
In IoT deployments, even nominally fully connected systems may have to 
deal with frequent loss of full connectivity.

\noindent\textbf{Vulnerability Analysis:}
Providing information about what devices can do makes it easier to 
automatically scan for devices with vulnerabilities.
An attacker may also use this information to plan attacks that take advantage of 
vulnerabilities in multiple devices. 
However, for the system manager, scanning can also be an opportunity 
to identify devices whose vulnerabilities need to be mitigated.


\noindent\textbf{Endpoint Adaptation:}
Metadata enables end-to-end security in networks of IoT devices using multiple standards.
If metadata is used to push payload adaptation to endpoints this allows 
communication payloads can be encrypted end-to-end.  
This contrasts with systems that use local bridging to connect devices from multiple IoT standards.
Local bridges require opening (and usually re-encrypting) data in potentially-vulnerable gateways.


\noindent\textbf{Secure Discovery:}
Information about how to use a service, 
and ideally even its existence, should not
be disclosed to agents without the authorization to use it.
The WoT approach allows powerful semantic searches to be used for discovery.
How can this capability be made available while still securing the metadata?


\noindent\textbf{Enabling Distributed Security:}
Metadata may be provided to enable specific security mechanisms,
as well as to support features with security implications such as payment or scripting.
What specific mechanisms are needed and what data needs to be provided
in order to satisfy the overall goal of interoperability?
Depending on how the metadata is made available, it may or may not be 
possible to support decentralized approaches to security.
In general, the mechanism used to provide the metadata is
an essential component of the security architecture.

The next few sections first introduce the W3C Web of Things draft standard,
focusing on the Thing Description metadata format.  
Then the high-level WoT Threat model will be introduced,
which includes a model of stakeholders, assets, attackers,
and threats, as well as a typical WoT deployment scenario.
Once this context has been established, we will discuss in detail the above
five issues.

