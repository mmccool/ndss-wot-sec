Many things must be taken into account when choosing what metadata to include in a TD: 
deployment scenarios and configurations, 
underlying networking protocols and their security mechanisms, 
overall scalability, system security priorities, and so on.
Security metadata should support security mechanisms in current use by both 
web services~\cite{openapi2018} and IoT systems~\cite{ace2017,ocf2017}.
The TD should be extensible enough to support emerging capabilities
related to security, for
example for micropayments~\cite{interledger2017} or smart contracts.

A data packet in a IoT system can go over many intermediate entities,
including gateways and proxies.
Some of these entities might need to perform authentication 
and authorization of the requester. 
Security metadata in a TD can indicate
what types of authentication are required by each entity
and how to obtain authorizations.
The metadata can also specify roles and security policies,
and perhaps different ones for different interactions.

Consider the deployment scenario in Figure~\ref{fig-wot-scenario}.
Suppose the forwarding proxy requires authentication of any request before passing it to the local network.
Additionally assume that the WoT gateway performs its own authentication.
Both of these authentication methods and 
associated information can be specified in the TD 
that the WoT Client receives from the Thing Directory.
Users of a WoT Thing may also have different roles with different access rights,
and may want to delegate some rights to others on a permanent or temporary basis.
%For example, a homeowner may have the ability to open and close a garage door and
%also query its current state and its power consumption.  The same homeowner may
%however want to allow a security service to monitor the state of the door in order
%to flag anomalies (such as the door opening when no one with access rights is
%at home). The homeowner may also want to be able to temporarily delegate rights
%to open the door to visitors.
%Yet another role is that of system administrator, who would need the ability to
%update the firmware on the device; such access would obviously be unwise to provide
%to visitors.

The WoT gateway can perform the authentication and access control to WoT Clients 
in a number of ways depending on what standards the target devices support.
For example, the OCF Security Specification~\cite{ocf2017}, which is one possible
system whose devices can be described by WoT TDs, 
supports a number of ways IoT devices can perform authentication based on 
either symmetrical or asymmetrical credentials (including certificates),
and also supports access control lists.
%However, provisioning credentials in advance 
%is not ideal for a distributed network such as the WoT is intended to support.
In general, we want to support fine-grained role-based access control
so that access rights can be granted (and revoked) without having to directly
update the state of devices.
Token-based authentication mechanisms
such as Bearer or Proof-of-Possession tokens~\cite{ace2017} can be used for this.
Token-based authentication allows  
a client to dynamically request a token from a remotely 
located authorization server and present it for 
authentication as needed.
This method has its own risks and limitations;
bearer tokens, for example, need to be protected from interception.
% However, token-based authentication can be combined with access control 
% lists to provide fine-grained role-based access policies.

Generalizing the above example, 
there might be $N$ sets of fully independent security metadata
necessary in a Thing Description.
Moreover, when a Thing Description is composed, 
these sets can be provided by separate entities: 
a gateway might only specify the security metadata for accessing interactions defined in a TD,
while the forwarding proxy adds the metadata required for successful authentication of incoming requests.
This means that there has to be a way to limit what security metadata 
is allowed to be provided by what entity.
Also, it must be possible for the WoT client to verify the overall integrity of the resulting TD. 
 
