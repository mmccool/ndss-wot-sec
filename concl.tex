We have given a summary of the W3C Web of Things draft
standard, with a focus on the Thing Description.
The Thing Description provides a descriptive approach to 
interoperability, which contrasts with the prescriptive
approach of most other standards.
While a prescriptive approach is useful, for example to
enforce minimum security requirements,
a descriptive approach can support brownfield devices
and can also make it easier to integrate devices that
conform to different prescriptive standards.

Beyond the basic issue of integrating devices
with different levels and mechanisms for security,
which will arise with any multistandard IoT system,
the use of descriptive metadata raises several
new security risks and opportunities.  

With the goal of starting discussion,
we have presented five such problems: 
security over local and multiple links,
multidevice vulnerability analysis,
end-to-end security enabling,
secure discovery for semantic interoperability,
and (potentially decentralized) security mechanism enabling.
