The WoT approach applies to a wide diversity of IoT devices and use cases.
The final WoT standard has to be able to support best practices in IoT security which
are however still rapidly evolving, and expected to keep on evolving.
Recently some attempts have been made to define
best practices for IoT security.
The IETF Thing-to-Thing Research group has a RFP under development,
\emph{State-of-the-Art and Challenges for the Internet of Things 
Security}~\cite{Garcia2017a}, which includes a threat model similar to 
what we have defined for the WoT.  
However, this threat model does not consider the 
importance of protecting access to descriptive metadata.
The Industrial Internet of Things has published a comprehensive
\emph{Security Framework}~\cite{Iic2016sf}.
This is useful, but focuses on industrial use cases.
The IoT Security Foundation has published
a \textit{Best Practices Guidelines}~\cite{Iotsf2017a}
document as well.
All three documents consider various additional factors we do
not have space to go into here, such as trust management over the 
lifecycle of the device.

There are several survey papers~\cite{Iot2020,Xu2014,Fernandes2017} that attempt to describe 
the current IoT security challenges and provide suggestions for future work in the area.

