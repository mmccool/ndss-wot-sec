Many issues arise when trying to access IoT devices
that are located on a ``local'' network,
for example behind a NAT in a Smart Home (consumer) use case,
or on a firewalled operational network in a Smart Factory
(industrial) use case.
In addition, devices may be accessible over multiple
routes and protocols, raising the problem of unique identification.

When one hears the term ``Web of Things'' the
assumption is that HTTP will be used to
access IoT devices. 
While leveraging web standards was one of the 
original goals of the Web of Things concept~\cite{Ostermaier2010},
the current WoT draft supports other choices as well,
such as CoAP and MQTT.
However, use of HTTP conveniently
allows Things acting as web servers to be accessed directly
from web browsers to provide human user interfaces via HTML.  
To provide better security, one would
naturally want to use HTTPS rather than HTTP.
Unfortunately the way HTTPS is supported in web browsers has
been designed for globally accessible web sites, not devices
behind NATs that may or may not have a globally accessible
address and may not even be continously connected to the internet.
In particular, certificate revocation checks as currently 
implemented in web browsers will not work if
the network both devices are on is not connected to the internet
(for example, if an \textit{ad hoc} network is used, with the device acting as
an access point) and certificates will not generally be able 
to tie the identity of a device to a particular unique URL. 

Both of these problems can be avoided by using a cloud proxy
or mirror (digital twin) for the device.
In that case a server
in the cloud is used as an intermediary.
Unfortunately,
this requires an active internet connection even to use 
local devices, has relatively high latency, and is
bandwidth-inefficient.
Various mechanisms have been explored to deal with this issue~\cite{httpslocal2017}
but no generally accepted solution has been adopted.
Therefore a metadata standard needs to be able to deal with a variety
of approaches.

% Other secure protocols, like CoAPS, 
% do not have the same assumptions as HTTPS and can be used
% more easily in an segmented network.
% As noted above the WoT, despite its name, can also work with these standards.
The issues noted above are not directly relevant if humans do not need
to communicate directly with a Thing using a web browser.  
In machine-to-machine (M2M) use cases
we have more flexibility in how we establish trust and validate connections.

Even for M2M use cases there is another issue: the URLs
used to access the same device can vary depending on whether the
device is accessible on the local network or should be accessed
via a global URL (either via NAT port forwarding, a proxy, or
a digital twin).
A Thing Description can include multiple links
for each interaction,
so in theory both local and global links
can be included in a single Thing Description.
However,
a user of a Thing has no easy way to tell if it is on
the same local network as another target Thing,
and if not, the ``local'' links won't work.
Another
approach would be for a trusted and authenticated
Thing Directory to return a modified
Thing Description with local links to clients that it (somehow) knows 
are on the same local network.
%However a mechanism is needed to ensure only
%authenticated and trusted Thing Directories can do such redirection,
%since redirection raises the possibility
%of a malicious Thing Directory, or a network attacker spoofing a
%Thing Directory, redirecting clients to fake devices.

A related issue is that of unique identity.
The linked data approach used to support semantics in the WoT
is based on the use of unique URIs to identify entities.
However, if a Thing can be accessed using multiple URLs then
these cannot be used to uniquely identify a Thing---or at least, not all of them. 
Currently under discussion are ways in
which unique identifiers can be added to Thing Descriptions,
separate from the links used for interactions.
These unique identifiers could be based on URI schemes such as UUIDs and DOIs.
Use of unique identifiers also raises (or perhaps only highlights) privacy considerations.
Potential privacy mitigations include limiting access to the unique identifiers in
Thing Descriptions to authorized users and
allowing unique identifiers to be reset as needed,
for example upon ownership transfer.
