\documentclass[conference]{IEEEtran}
\pagestyle{plain}
\hyphenation{op-tical net-works semi-conduc-tor}

\usepackage{todonotes}
\usepackage[tight,footnotesize]{subfigure}

\usepackage{hyperref}
\hypersetup{
	breaklinks,
    pdfpagemode={UseOutlines},
    bookmarksopen,
    pdfstartview={FitH},
    colorlinks,
    linkcolor={blue},
    citecolor={red},
    urlcolor={blue}
}


\begin{document}
\title{
Distributed Security Risks and Opportunities \\
in the %\\
W3C Web of Things
}
\author{
  \IEEEauthorblockN{Michael McCool}
  \IEEEauthorblockA{Intel Corporation\\
                    michael.mccool@intel.com}
\and
  \IEEEauthorblockN{Elena Reshetova}
  \IEEEauthorblockA{Intel Corporation\\
                    elena.reshetova@intel.com}
}

\IEEEoverridecommandlockouts
\makeatletter\def\@IEEEpubidpullup{9\baselineskip}\makeatother
\IEEEpubid{\parbox{\columnwidth}{Permission to freely reproduce all or part
    of this paper for noncommercial purposes is granted provided that
    copies bear this notice and the full citation on the first
    page. Reproduction for commercial purposes is strictly prohibited
    without the prior written consent of the Internet Society, the
    first-named author (for reproduction of an entire paper only), and
    the author's employer if the paper was prepared within the scope
    of employment.  \\
    NDSS '18, 18-21 February 2018, San Diego, CA, USA\\
    Copyright 2018 Internet Society, ISBN 1-1891562-49-5\\
    http://dx.doi.org/10.14722/ndss.2018.23008
}
\hspace{\columnsep}\makebox[\columnwidth]{}}
\maketitle

\begin{abstract}
The W3C Web of Things (WoT) WG has been developing an interoperability standard for IoT devices that includes as
its main deliverable a ``Thing Description'': a standardized representation the metadata of an 
IoT device, including in particular its network interface, but also allowing for semantic annotation.
Relative to other approaches to IoT, such metadata has at least four major implications.
First, it allows for system-wide vulnerability analysis, 
which can be both a risk and an opportunity.
Second, metadata can enable end-to-end security in multistandards networks,
avoiding exposing data within bridges otherwise needed for connecting standards pairwise.
Third, metadata supports service and device discovery,
which raises the question of how to limit discovery to authorized agents.
Fourth, metadata can enable distributed security mechanisms for access control and micropayments.
To the extent that metadata access can be decentralized, decentralized mechanisms for security can
be supported, although several practical issues currently make this difficult to fully support.

\end{abstract}

% Introduction: state main goals and context of paper
\section{Introduction}
The economic impact of the IoT will be strongly dependent on how well devices from 
different manufacturers can interoperate.
Very often wide interoperability is taken for granted when estimating the business
benefit of IoT. 
Conversely, if IoT devices do \emph{not} interoperate,
a recent study~\cite{McK2015a} concluded that 40\% to 60\% of the 
benefit of IoT will be unattainable,
due to the inability to address use cases that cannot be satisfied by a single manufacturer.

Full interoperability is hard to achieve.
There are currently many competing IoT standards under development,
each of which is attempting to address interoperability.
Most of these standards are prescriptive.
In a prescriptive standard,
devices are validated against specific requirements.
Typically the goal is that devices validated against 
a particular standard will interoperate with
other devices also validated against that standard.
It is also possible to bridge multiple standards so that
devices validated against one standard can communicate with
devices using another standard by translating communication protocols and payloads.
If one standard comes to dominate bridging will be unnecessary,
but so far such unification seems to be elusive. 
Unification is complicated by the
divergent requirements in different but overlapping IoT subdomains.
As an example of divergent requirements,
some application areas require real-time response (bounded low-latency responses)
and others require low power (requiring long sleep times) or wireless access
(in which it is very difficult to acheive real-time guarantees due to the
possibility of interference or collisions).
Basic interaction patterns can also vary: some standards use
a web-inspired resource-centric (RESTful) client/server interaction model,
while others use a message-centric publish-subscribe model.

%Unfortunately the prescriptive approach has some weaknesses.
In addition to the issue of interoperability between modern prescriptive IoT standards,
there are always going to exist devices that follow older standards or specifications.
There are decades-old devices in particular domains, such as building and factory
automation, that are just now being connected to the IoT.
These devices often represent major investments and cannot be economically replaced with newer
devices conforming to the latest standard.
This is the ``brownfield'' problem.
Moreover new devices are being deployed today and for the forseeable future 
that have not been validated
against any particular IoT standard,
even if they use other \textit{de jure} or \textit{de facto} standards such as JSON, HTTP, or MQTT.

As an alternative to the prescriptive approach,
the W3C Web of Things (WoT) Working Group has been developing a \emph{descriptive} 
approach to IoT interoperability.
In the descriptive approach,
metadata is provided that describes how to communicate with each particular device,
using a set of interaction patterns that includes as a subset both resource-centric
and message-centric interaction models.
The metadata itself is standardized but flexible enough to describe a wide variety of
IoT network interfaces.
With this approach,
devices can but do not have to be prevalidated against 
a particular standard before being deployed.
They can be described after the fact,
and do not need any modification to be
used as part of a Web of Things system.
This solves the brownfield problem and allows
older devices as well as devices satisfying different IoT 
standards to be integrated into a unified system.  

This approach has both risks and opportunities from a security point of view.
Most obviously, IoT devices, even those conforming to a prescriptive IoT standard,
may vary widely in their support for security.
Therefore a Web of Things system
needs to manage different levels of trust for different devices.
Devices from different ecosystems or manufacturers may also take different approaches to
security, have different trust models, have different levels of acceptable risk,
and may use different security mechanisms. 
This may cause integration challenges, even if the necessary
information is provided in the metadata.

% Beyond this basic concern,
% the availability of pervasive metadata raises several other issues
% from a security perspective.
% In this paper we discuss five major issues:
% NOTE: want to use {description} here but it seems to be broken...
It is vital that new standards are analyzed for security risks before,
not after, deployment.
The goal of this paper is to start the discussion of
the risks and opportunities presented
by WoT metadata and, 
more generally, the descriptive approach to interoperability.  
\textit{\textbf{We do not present solutions, but rather a set of problems.}}
We have identified at least five such problems:

\noindent\textbf{1.~Local Links:}
End-point IoT devices can be deployed in many ways: 
starting with deployments in closed, only locally accessible,
networks, extending to systems on local networks but behind a proxy or a firewall 
providing access to the global internet, and ending with deployments on a globally addressed network. 
In fact, it may be possible to reach the same device via multiple paths.
As a result the metadata provided by IoT devices should allow for expression of
different ways (links) to reach each device 
and a way to securely update this information when the deployment setup changes. 
Security mechanisms with assumptions about global connectivity also may not
operate correctly in disconnected networks.
In IoT deployments, even nominally fully connected systems may have to 
deal with frequent loss of full connectivity.

\noindent\textbf{2.~Vulnerability Analysis:}
Providing information about what devices can do makes it easier to 
automatically scan for devices with vulnerabilities.
An attacker may also use this information to plan attacks that take advantage of 
vulnerabilities in multiple devices. 
However, for the system manager, scanning can also be an opportunity 
to identify devices whose vulnerabilities need to be mitigated.

\noindent\textbf{3.~Endpoint Adaptation:}
Metadata enables end-to-end security in networks of IoT devices using multiple standards.
Pushing payload adaptation to endpoints is an enabler for
end-to-end encryption of payloads via object security or tunnelled transports (or both).
This contrasts with systems that use local bridging to connect devices from multiple IoT standards.
Local bridges require opening (and usually re-encrypting) data in potentially-vulnerable gateways.

\noindent\textbf{4.~Secure Discovery:}
Information about how to use a service, 
and ideally even its existence, should not
be disclosed to entities without the authorization to use it.
The WoT approach allows powerful semantic searches to be used for discovery.
How can this capability be made available while still securing the metadata?

\noindent\textbf{5.~Enabling Distributed Security:}
Metadata may be provided to enable specific security mechanisms,
as well as to support features with security implications such as payment or scripting.
What specific mechanisms are needed and what data needs to be provided
in order to satisfy the overall goal of interoperability?
Depending on how the metadata is made available, it may or may not be 
possible to support decentralized approaches to security.
In general, the mechanism used to provide the metadata is
an essential component of the security architecture.

The next few sections first introduce the W3C Web of Things draft standard,
focusing on the Thing Description metadata format.  
Then the high-level WoT Threat model will be introduced,
which includes a model of stakeholders, assets, attackers,
and threats, as well as a typical WoT deployment scenario.
Once this context has been established, we will discuss in detail the above
five issues.



% Explain background (in this case, WoT) 
\section{Web of Things}
% Nowadays the amount of different IoT devices around us is 
% constantly increasing in all areas starting from smart home and ending up with industry automation. 
% All these devices target different use cases,
% have different underlying implementations and connected using different communication protocols.
% In this light 
The Web of Things (WoT)~\cite{Wot2017arch} aims to provide interoperability between IoT devices.
It does this by defining a metadata format,
the WoT \emph{Thing Description}, that can describe 
a wide range of IoT network interfaces.
A Thing Description can describe the network interfaces of existing devices
or can be produced and consumed by devices running a WoT runtime supporting a WoT scripting
API that normalizes interactions with other devices with a common abstraction layer.


\subsection{Architecture}
%\textsf{Web of Things architecture \cite{Wot2017arch}, Thing Description \cite{Wot2017td} and
%scripting API \cite{Wot2017script}.}
The Web of Things (WoT) architecture\cite{Wot2017arch} defines three basic entities
that can be organized into various configurations and topologies 
based on a concrete deployment scenario:

A \textbf{WoT Thing} represents a physical or virtual IoT device 
        and exposes a network-facing API for interaction.
	Each WoT Thing has an associated Thing Description (TD)\cite{Wot2017td}. 
        A TD encodes a set of metadata describing relevant information about a Thing,
        such as semantic categorization, available interactions, and communication and security mechanisms.
	A typical example of a WoT Thing might be a garage door controller.
        Such a controller would provide a number of interactions that can be performed on a garage door, 
        i.e. \textit{open}, \textit{close}, etc. and would provide network interfaces to invoke
        each of these.
        Typically a WoT Thing plays the network role of a server as it responds to
        but does not initiate interactions.


A \textbf{WoT Client} is an entity that wants to perform an action on a WoT Thing.
	It is able to consume a TD provided by (or for) a WoT Thing and invoke interactions on 
        the target's network interfaces.
	For example a WoT Client might be a browser or an application on a user's smartphone
        that allows the user to invoke one of the interactions provided by the garage door controller. 


A \textbf{WoT Servient} can be viewed as a combination of a client and server:
        an entity that is both providing one or more WoT Thing interfaces (as a server) and
        at the same time is able to operate as WoT Client to invoke interactions on other WoT Things.
	An example of a WoT Servient would be (a service running on a) home gateway device 
        that acts as a WoT Client towards home appliance WoT Things
        (such as different lights and sensors) and also exposes some higher level
        virtual devices (such as the collection of all the lights in the living room)
        in form of additional WoT Things available for a WoT Client running on a user's smartphone.

Internally a typical architecture of a WoT Servient is shown in Figure~\ref{fig-fservient}. 
In addition to a Thing Description (TD) it also has WoT Binding Templates 
that can be used to instantiate a TD for a particular IoT protocol binding, 
such as OCF, HTTP(S), COAP etc. 
Internally a WoT Servient can also host a WoT Runtime and
a WoT Scripting API.
The WoT Scripting API is an optional component that allows 
implementing logic of an application Servient in a standardized way
using a higher-level programming language (such as JavaScript). 

\begin{figure}[!t]
\centering
\includegraphics[width=4in]{figures/wot-servient.png}
\caption{WoT Servient architecture}
\label{fig-fservient}
\end{figure}




\subsection{Threat Model}
Due to the large diversity of devices,
use cases,
deployment scenarios 
and requirements for WoT Things,
it is impossible to define a single WoT threat model
that would fit every use case.
Instead we have created an overall WoT threat model
\emph{framework}~\cite{Wot2017sec} 
that can be adjusted by OEMs or other WoT system users or providers
based on their concrete security requirements.

%The WoT threat model defines the following key WoT Assets that are important from security point of view:	\textit{Thing Description}, \textit{Thing User Data} (any user data transferred via WoT network, such as video streams, sensor data etc.), \textit{Thing Provider Data} (WoT application scripts and their configuration data), \textit{WoT Basic Security Metadata} (all provisioned security medatada), \textit{WoT Controlled Environment} (physical environment that can be affected by WoT Things), \textit{WoT Thing and Infrastructure Resources} (resources of devices providing WoT Things and overall WoT network) and \textit{WoT Behavior Metrics} (all indirectly transmitted information).
%Some of these assets might be absent and/or have different trust model (i.e who should have a legitimate access and to what extent) depending on deployment scenario. 

The threat model defines security-relevant WoT assets 
and a set of threats on these assets.
Threats can be in or out of scope based on the deployment scenario,
security objectives, acceptable risks, and other factors.
For example,
in a smart home scenario involving WoT Things that record audio/video
information, privacy would be considered important and 
therefore this scenario would have high confidentiality requirements.
On the other hand,
in an industry automation scenario involving WoT Things that 
control some safety-critical infrastructure,
service availability, access control,
and protection of the equipment and environment controlled by the Thing 
may be of the most importance, not privacy.

Deploying the WoT in a distributed system brings additional complexity 
to the WoT Threat Model and the choice of relevant security mitigations.
In particular we cannot rely on single standard communication infrastructure 
and protocol 
(like HTTP/TLS) to provide end-to-end security between all communicating
parties.
Instead we need to support multiple security mechanisms 
(potentially nested or interdependent) 
that must be somehow combined into a unified security solution.  
 


\subsection{Typical Deployment Scenario}
Figure~\ref{fig-wot-scenario} presents a typical WoT deployment scenario. 
A WoT Thing device together with a WoT Servient are located on a 
local network and accessed through a forwarding proxy. 
A WoT Client (which may in general be located either outside of the local network,
as shown, but in many use cases may also be inside) wants to 
perform interactions on the WoT Thing and WoT Servient.
The available interactions are given in corresponding Thing Descriptions.
There are various ways Thing Descriptions provided by WoT Things
could be made available to WoT Clients.
In this case, the Thing Descriptions are 
uploaded by the WoT Thing and WoT Servient to a Thing Directory.
A Thing Directory is a service which can be accessed by Clients
to search for WoT Things it can communicate with.
In order to do this the WoT Client first needs to issue a 
discovery query to the Thing Directory.
Thing Directories can support semantic search capabilities, allowing
Things to be discovered based on semantic annotations. 
Access to the search capabilities of a Thing Directory should
only be provided to authorized clients, as with any web service.
Upon obtaining a Thing Description, 
the WoT Client needs to make sure it has all the necessary credentials
to authenticate to the forwarding proxy 
(assuming secure authentication on the proxy is enabled), 
the WoT Servient and in some cases even to the specific WoT Thing.
Information describing how clients need to authenticate themselves
should be provided in the obtained Thing Descriptions.  

\begin{figure*}[!t]
\centering
\includegraphics[width=6in]{figures/wot-scenario.png}
\caption{Typical WoT deployment scenario}
\label{fig-wot-scenario}
\end{figure*}


% Best practices in IoT that are *common* to WoT... and that we will NOT focus on
% We want to explain in particular how WoT is different
\section{Related Work}
The WoT approach applies to a wide diversity of IoT devices and use cases.
The final WoT standard has to be able to support best practices in IoT security which
are however still rapidly evolving, and expected to keep on evolving.
Recently some attempts have been made to define
best practices for IoT security.
The IETF Thing-to-Thing Research group has a RFP under development,
\emph{State-of-the-Art and Challenges for the Internet of Things 
Security}~\cite{Garcia2017a}, which includes a threat model similar to 
what we have defined for the WoT.  
However, this threat model does not consider the 
importance of protecting access to descriptive metadata.
The Industrial Internet of Things has published a comprehensive
\emph{Security Framework}~\cite{Iic2016sf}.
This is useful, but focuses on industrial use cases.
The IoT Security Foundation has published
a \textit{Best Practices Guidelines}~\cite{Iotsf2017a}
document as well.
All three documents consider various additional factors we do
not have space to go into here, such as trust management over the 
lifecycle of the device.

There are several survey papers~\cite{Iot2020,Xu2014,Fernandes2017} that attempt to describe 
the current IoT security challenges and provide suggestions for future work in the area.



% What are some specific interesting problems/solutions that WoT brings?
\section{Risks and Opportunities}
\label{sec-risks-opportunities}

\subsection{Local Links}
Practical pitfalls.  HTTPS not working locally.  Local links vs global links in Thing Directories.


\subsection{Vulnerability Scanning}
\textsf{Vulnerability scanning using metadata.}


\subsection{Endpoint Adaptation}
\textsf{End-to-end secure adaptation by pushing payload transformation to endpoints.}


\subsection{Secure Discovery}
\textsf{Secure semantic searches.  How do we ensure only the data permitted for a user is used in a search?}
Some possibly relevant papers: \cite{Thura2005,Xia2014}.


\subsection{Enabling Distributed Security}
Choosing what security metadata to include in a TD is not a straightforward task.
Many things must be taken into account: 
various deployment scenarios and configurations, 
underlying networking protocols and their security mechanisms, 
overall scalability, system security priorities, and so on.

A data packet in a WoT network can go over many intermediate entities,
including gateways and proxies.
Some of these entities might need to perform authentication 
and authorization of the requester. 
The security metadata in the TD should indicate
what types of credentials are required by each entity and how to obtain them.

For example, consider the deployment scenario in Figure~\ref{fig-wot-scenario}.
Suppose the forwarding proxy requires authentication of any request before passing it to the local network.
Additionally assume that the WoT gateway performs authentication 
at least for certain critical interactions provided by the WoT Thing 
(for example, unlocking a door). 
Both of these authentication methods (potentially different) and 
associated information must be specified in the TD 
that the WoT Client receives from the Thing Directory.
The WoT gateway can perform the authentication of the WoT Client 
in a number of ways depending on what standards the target devices support.
For example, the OCF Security Specification~\cite{ocf2017} 
describes a number of ways IoT devices can perform authentication based on 
either symmetrical or asymmetrical credentials (including certificates). 
However, provisioning credentials in advance 
is not ideal for a distributed network such as the WoT is intended to support.
Another possibility is to use a token-based authentication mechanism,
such as OAuth 2.0 or Proof-of-Possession tokens~\cite{ace2017}.
Token-based authentication allows  
a WoT Client to dynamically request a token from a remotely 
located authorization server and present it for 
authentication to a WoT gateway or Thing.
This method has its own risks, as tokens need to be protected
from interception.

Generalizing the above example, 
there might be $N$ sets of fully independent security metadata
necessary in a Thing Description.
Moreover, when a Thing Description is composed, 
these sets can be provided by separate entities: 
a gateway might only specify the security metadata for accessing interactions defined in a TD,
while the forwarding proxy adds the metadata required for successful authentication of incoming requests.
This means that there has to be a way to limit what security metadata 
is allowed to be provided by what entity.
Also, it must be possible for the WoT client to verify the overall integrity of the resulting TD. 
 


\section{Conclusion}
\textsf{Conclusions.  What are the main points?}


\section*{Acknowledgment}
The security and threat model presented here was developed 
in the W3C Web of Thing WG as part of the 
\textit{Web of Things (WoT) Security and Privacy Considerations} 
document~\cite{Wot2017sec}. Please see the associated github site
for a list of additional contributors.
Barry Leiba provided feedback on an early draft.


\bibliographystyle{IEEEtranS}
\bibliography{refs}
\end{document}


