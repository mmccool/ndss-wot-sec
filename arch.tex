The Web of Things (WoT) architecture~\cite{Wot2017arch} defines three basic entities
that can be organized into various configurations and topologies 
based on a concrete deployment scenario:

\noindent\textbf{$\bullet$ WoT Thing:} A software entity
        representing a physical or virtual IoT device 
        that exposes a network-facing API for interaction.
	Each WoT Thing has an associated Thing Description (TD)~\cite{Wot2017td}. 
        A TD encodes a set of metadata describing relevant information about a Thing,
        such as semantic categorization, available interactions, and communication and security mechanisms.
        Typically a WoT Thing plays the network role of a server that responds to
        but does not initiate interactions.
\emph{For example:} 
A WoT Thing might be a garage door controller.
Such a controller would provide a number of interactions that can be performed on a garage door, 
i.e. \textit{open}, \textit{close}, etc. and would provide network interfaces to invoke
each of these.

\noindent\textbf{$\bullet$ WoT Client:} An entity that can operate on a WoT Thing.
It is able to consume a TD provided by (or for) another WoT Thing and invoke interactions on 
its network interfaces.
\emph{For example:} 
A WoT Client might be a browser or an application on a user's smartphone
that allows the user to invoke one of the interactions provided by the above garage door controller. 

\noindent\textbf{$\bullet$ WoT Servient:} An entity that can be viewed as a combination of a client and server.
        A servient both provides one or more WoT Thing interfaces (as a server) and
        at the same time is able to operate as a WoT Client to invoke interactions on other WoT Things.
\emph{For example:} 
A WoT Servient might be a service running on a home gateway device 
that provides a generalized ``lock up'' service (including closing the garage door, but also 
arming the home alarm, securing door locks, turning off lights, etc.) with its own network interface.

Internally the typical architecture of a WoT Servient is shown in Figure~\ref{fig-fservient}. 
A WoT Servient supports one or more Thing Descriptions (TDs)
and associated protocol binding templates.
The binding templates are used to instantiate a TD for a particular IoT protocol, 
such as OCF, HTTP, MQTT, CoAP etc.

A WoT Servient can also host a WoT Runtime and a WoT Scripting API.
The WoT Scripting API is an optional component that allows 
implementing the logic of an application Servient in a standardized way
using a higher-level programming language (the current WoT draft focuses on JavaScript). 
In this document, however, we focus on the security implications of the metadata alone,
and do not consider further the security implications of the Scripting API (which
raises many additional security issues such as application isolation,
prevention of DDoS attacks, and so on).

\begin{figure}[!t]
\centering
\includegraphics[width=3.3in]{figures/wot-servient.png}
\caption{WoT Servient architecture}
\label{fig-fservient}
\end{figure}


