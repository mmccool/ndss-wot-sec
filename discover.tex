One of the benefits provided by the WoT approach is to
enable the use of powerful semantic searches during discovery.
However, this introduces several issues related to security.

First of all, semantic searches can be abused to create DoS attacks.
If arbitrary SPARQL endpoints are provided by Thing Directory
services, semantic searches can be specified
that can consume large amounts of resources,
rendering the Thing Directories unavailable to other users.
This can be mitigated by either limiting the power of searches
that can be specified, or by limiting the processing
power that can be used in a specific search.

In the first approach, rather than providing a full
SPARQL endpoint, a directory service may provide a more specialized
search interface that only allows searches for a conjunction of terms
and does not allow, for instance, use of arbitrary inference rules
or use of other SPARQL endpoints in a federated search.
However, by limiting the power
of the searches that may be specified, we are also limiting the
potential benefit.

The second approach to mitigate DoS attacks 
is to use an architecture for the 
search engine that can enforce resource quotas on individual queries.
For example, each search could be spawned in a new process and Linux
process limits could be used to enforce processing quotas. In addition,
the number of searches that can be processed at the same time would
have to be limited to avoid creating too many processes.
If either limit is exceeded the server would return an appropriate
error code.  If the query processing quota is exceeded, the client
would have to reformulate their query.  If the server is too busy,
the client would have to retry the query later.

Of course creating a new process for each search would be expensive,
so ideally the search engine itself would have lighter-weight
mechanisms to track resource consumption and enforce quotas.

There is another class of risks with semantic searches: 
inferencing.  Semantic searches can infer information that is
not explicitly present in the database: that is point!
However, it is possible that an attacker could use this capability
to invade someone's privacy.  For example, they could infer
personal information, such as marital status, from the ownership
of certain combinations of devices (for example, multiple toothbrushes).
While it is difficult to prevent inference attacks in general,
we should at least design semantic search engines to not use 
data for inference that would not be directly available to the
person posing the query in the first place~\cite{Thura2005a,Xia2014a}.  
As with the other class of risks we mentioned,
mitigation requires restricting the distribution of Thing Description
data to authorized users only.
