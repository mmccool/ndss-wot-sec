%\textsf{A basic intro to to the WoT threat model~\cite{Wot2017sec}.}

Due to the large diversity of devices,
use cases,
deployment scenarios 
and requirements for WoT Things,
it is impossible to define a single WoT threat model
that would fit every case.
Instead we have created an overall WoT threat model
\emph{framework}~\cite{Wot2017sec} 
that can be adjusted by OEMs or other WoT system users or providers
based on their concrete security requirements.

%The WoT threat model defines the following key WoT Assets that are important from security point of view:	\textit{Thing Description}, \textit{Thing User Data} (any user data transferred via WoT network, such as video streams, sensor data etc.), \textit{Thing Provider Data} (WoT application scripts and their configuration data), \textit{WoT Basic Security Metadata} (all provisioned security medatada), \textit{WoT Controlled Environment} (physical environment that can be affected by WoT Things), \textit{WoT Thing and Infrastructure Resources} (resources of devices providing WoT Things and overall WoT network) and \textit{WoT Behavior Metrics} (all indirectly transmitted information).
%Some of these assets might be absent and/or have different trust model (i.e who should have a legitimate access and to what extent) depending on deployment scenario. 

The threat model defines security-relevant WoT assets 
and a set of threats on these assets.
Threats can be in or out of scope based on the deployment scenario,
security objectives, acceptable risks, and other factors.
For example,
in a smart home scenario involving WoT Things that record audio/video
information, privacy would usually be considered important and 
therefore this scenario would have high confidentiality requirements.
On another hand,
in an industry automation scenario involving WoT Things that 
control some safety-critical infrastructure,
confidentiality might not be the highest priority.
Instead, in an industrial setting, service availability
and protection of the environment controlled by the Thing 
may be of topmost importance.

Deploying the WoT in a distributed system brings additional complexity 
to the WoT Threat Model and the choice of relevant security mitigations.
In particular one cannot rely on single standard communication infrastructure 
and protocol 
(like HTTPS) to provide an end-to-end security between all communicating
parties.
Instead WoT needs to support multiple security mechanisms 
(potentially nested or interdependent) 
that can be combined into single end-to-end security solution.  
 
