The W3C Web of Things (WoT) WG has been developing an interoperability standard for IoT devices that includes as
its main deliverable a ``Thing Description'': a standardized representation for the metadata of an 
IoT device, including in particular a description of its network interface,
but also allowing for multiple levels of semantic annotation.
The WoT Thing Description supports a descriptive (as opposed to prescriptive) approach to interoperability.
The provision of rich descriptive metadata has at least five major implications for security.
First, the need for local links and, more generally, disconnected and segmented networks
       in IoT raises several practical considerations in what metadata should be provided.
Second, metadata allows for system-wide vulnerability analysis, 
       which can be both a risk and an opportunity.
Third, metadata can enable end-to-end security in multistandards networks,
       avoiding exposing unencrypted data within bridges otherwise needed for connecting standards pairwise.
Fourth, metadata supports service and device discovery,
       which raises the question of how to limit discovery to authorized agents.
Fifth, metadata can enable distributed security mechanisms for access control and micropayments.
       To the extent that metadata access can be decentralized, decentralized mechanisms for security can
       be supported.
