The W3C Web of Things (WoT) WG has been developing an interoperability standard for IoT devices that includes as
its main deliverable a ``Thing Description'': a standardized representation the metadata of an 
IoT device, including in particular its network interface, but also allowing for semantic annotation.
In this paper, we discuss the implications of standardized metadata on security.
Standardized metadata has both risks and opportunities.
On the one hand, information about what devices can do makes it easier to scan for devices with 
vulnerabilities.  However, this risk can in fact be an opportunity as scanning for devices with
vulnerabilities is necessary to identify devices whose vulnerabilities need to be mitigated.

Pervasive metadata has one major additional benefit: it enables end-to-end security in multistandards networks.
Specifically, if metadata is used to push payload adaptation to endpoints then the need to unpack and
reformat data in gateways can be eliminated, and payloads can be encrypted end-to-end.  This contrasts
with systems that use local bridging between multiple IoT standards which requires opening (and usually
deencrypting) data in potentially-vulnerable gateways.

The metadata also needs to be secured.  Thing descriptions can be delivered from devices themselves, or
from directory services. Directory services can also support discovery, including discovery based on
semantic search.  This raises the problem of how discovery can be constrained to return data that
the searcher is authorized to access.
