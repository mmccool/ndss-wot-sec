Figure~\ref{fig-wot-scenario} presents a typical WoT deployment scenario. 
A WoT Thing device together with a WoT Servient are located on a 
local network and accessed through a forwarding proxy. 
A WoT Client (which may in general be located either outside of the local network,
as shown, but in many use cases may also be inside) wants to 
perform interactions on the WoT Thing and WoT Servient.
The available interactions are given in corresponding Thing Descriptions.
There are various ways Thing Descriptions provided by WoT Things
could be made available to WoT Clients.
In this case, the Thing Descriptions are 
uploaded by the WoT Thing and WoT Servient to a Thing Directory.
A Thing Directory is a service which can be accessed by Clients
to search for WoT Things it can communicate with.
In order to do this the WoT Client first needs to issue a 
discovery query to the Thing Directory.
Thing Directories can support semantic search capabilities, allowing
Things to be discovered based on semantic annotations. 
Access to the search capabilities of a Thing Directory should
only be provided to authorized clients, as with any web service.
Upon obtaining a Thing Description, 
the WoT Client needs to make sure it has all the necessary credentials
to authenticate to the forwarding proxy 
(assuming secure authentication on the proxy is enabled), 
the WoT Servient and in some cases even to the specific WoT Thing.
Information describing how clients need to authenticate themselves
should be provided in the obtained Thing Descriptions.  

\begin{figure*}[!t]
\centering
\includegraphics[width=6in]{figures/wot-scenario.png}
\caption{Typical WoT deployment scenario}
\label{fig-wot-scenario}
\end{figure*}
